\documentclass[11pt]{article}
\setlength{\oddsidemargin}{.25in}
\setlength{\evensidemargin}{.25in} \setlength{\textwidth}{6in}
\setlength{\topmargin}{-0.4in} \setlength{\textheight}{8.5in}
\def\argmax{\mathop{\rm arg\,max}}

\usepackage{xspace,epsfig,amsmath,amssymb,lmodern}
\usepackage{booktabs}
\usepackage[all]{xy}

\newtheorem{theorem}{Theorem}
\newtheorem{lemma}{Lemma}
\newtheorem{definition}{Definition}
\newtheorem{corollary}{Corollary}
\newtheorem{proposition}{Proposition}

\newenvironment{sketch}{\noindent\emph{Proof Sketch:}}{$\quad \Box$}
\newenvironment{proof}{\noindent\emph{Proof:}}{$\quad \Box$}
\newcommand{\handout}[5]{
   \renewcommand{\thepage}{\arabic{page}}
   \noindent
   \begin{center}
   \framebox{
      \vbox{
    \hbox to 5.78in { {\bf Choice Models in Operations}  \hfill #2
}
       \vspace{4mm}
       \hbox to 5.78in { {\Large \hfill #5  \hfill} }
       \vspace{2mm}
       \hbox to 5.78in { {\it #3 \hfill #4} }
      }
   }
   \end{center}
   \vspace*{4mm}
}

\begin{document}

\handout{}{}{Instructor: Srikanth Jagabathula}
{Scribe:  Sherry Wang } {Lecture
  1. :  Preference Theory}
\section{Notation}
\begin{table}[htbp]
  \centering
  \caption{Notation}
    \begin{tabular}{ccc}
    \toprule
       Calliqraphic symbols   & Sets & $\mathcal{X,N,M}$ \\
    Sans seril font   & Random variables & $\mathsf{U}, \mathsf{X}, \mathsf{Y}$ \\
    Capital letter & Matrices & $A,B$\\
    Greek and small letters & cardinality, indexing & $i,j,m,n$\\
    \bottomrule
\end{tabular}
\end{table}
\section{Preference Theory}
$\mathcal{N}$: universe of $n$ items\\
$2^{\mathcal{N}}$: power set of $\mathcal{N}$, the set of collection of all subsets of $\mathcal{N}$.\\
\subsection{Choice rule}
\begin{definition}
  A function $C$: from $2^{\mathcal{N}}$ to $2^{\mathcal{N}}$, s.t. $C(\mathcal{B})\subseteq \mathcal{B}, \forall \mathcal{B} \in 2^{\mathcal{N}}$.\\
  If $x\in C(\mathcal{B})$, then we say that ``$x$ is chosen from $\mathcal{B}$".\\
  If $y \in \mathcal{B}$, then we say that ``$y$ could have been chosen from $\mathcal{B}$"
\end{definition}
In classical economics, $\mathcal{B}$ is often called ``budget set". In operation and marketing, $\mathcal{B}$ is an ``offer set'' or ``choice set''.\\
Remarks:
\begin{enumerate}
  \item $C(\mathcal{B})$ doesn't have to be a singleton.
  \item $C(\mathcal{B})$  may be empty.
  \item If $C(\mathcal{B})\neq \phi, \forall \mathcal{B} \in 2^{\mathcal{N}}$, then $C(.)$ is ``decisive''.
  \item $C(.)$ is a singleton $\forall \mathcal{B}$, then $C(.)$ is ``univalent''.
\end{enumerate}

\subsection{Preference rule}
\begin{definition}
We define a weak preference $\succeq$ of a customer over the set $\mathcal{N}$ as follows:
$x \succeq y \iff x$ is at least as good as $y$.
``Making elements of $\mathcal{N}$ comparable to each other''
\end{definition}
\begin{definition}
We say that $x$ is strictly preferred to $y$ if $x \succeq y $ but $y \nsucceq x$.
\end{definition}

We make 2 assumptions:
\begin{enumerate}
\item Completeness: A weak preference $\succeq$ is complete iff $\forall x, y \in \mathcal{N}$ we have either $x \succeq y $ or $y \succeq x$, or both.
\item Transitivity: A preference relation is transitive if whenever $x \succeq y $ and $y \succeq z $ then $x \succeq z, \forall x,y,z \in \mathcal{N}$
\end{enumerate}
Representation: we can represent a rational preference relation as a graph with the items as nodes and a directed edge from $x$ to $y$ iff $x \succeq y $.
\begin{displaymath}
 \xymatrix{
1 \ar[r] &2\ar[d]\\
3  \ar[u] &4\ar[l]}
\end{displaymath}
In this graph, you can cycle among all the products.
Kahneman and Tversky
Eg: Participants want to buy \$125 stereo and \$15 calculator.\\
Q1: You learn that there is a \$5 discount on calculator at another store. Will you go to the other store?\\
Q2: You learn that there is a \$5 discount on stereo at another store. Will you go to the other store?\\
Q3: Both are out of stock. You are offered a \$5 discount to go to the other store. Do you care which item is discounted?\\
Most people say YES to Q1, NO to Q2, INDIFFERENT to Q3.\\
$x$: travel to the other store for discount on calculator.\\
$y$: travel to the other store for discount on stereo.\\
$z$: stay at the store.\\
Q1: YES implies $x \succ z$\\
Q2: NO implies $z \succ y$\\
By rationality, $x \succ y$\\
Q3: INDIFFERENCE implies $x \sim y$\\

\begin{definition}
Every preference relation $\succeq$ that is rational induces a choice rule: $C(\mathcal{B};\succeq)=\{x\in \mathcal{B}|x\succeq y, \forall y \in \mathcal{B}\}$
\end{definition}
Remarks:
\begin{enumerate}
\item $C(\mathcal{B};\succeq)$ may contain more than one element.
\item $C(\mathcal{B};\succeq)$ may be empty, only if the sets are infinite.\\
e.g. $\succeq$ is $\geq$, let $\mathcal{N}=R, \mathcal{B}=[0,1), then \{x \in \mathcal{B}:x\succeq y, \forall y \in \mathcal{B} \}=\phi$\\
If $\mathcal{N}$ is finite, then $C(\mathcal{B};\succeq)$ is always nonempty.
\end{enumerate}
\begin{definition}
HARP:Houthaker's Axiom of Revealed Preference\\
A choice function satisfies HARP if whenever $x,y \in \mathcal{A}\cap \mathcal{B}$ and $x\in C(\mathcal{A}), y \in C(\mathcal{B})$, then we must have $x\in C(\mathcal{B}), y \in C(\mathcal{A})$
\end{definition}

%\begin{proposition}
%  Suppose there exists a complete and transitive relation $\succeq$ on $\mathcal{N}$, s.t. $C(.)=C(.;\succeq)$ iff $C(.)$ satisfies HARP.
%\end{proposition}

\begin{theorem}
  C(.) is non empty and satisfies HARP iff $\exists$ a complete and transitive $\succeq$ s.t. $C(\mathcal{B})=C(\mathcal{B};\succeq), \forall \mathcal{B}$
\end{theorem}
\begin{proof}
  $\Longleftarrow$    \\
  We first prove that $C(.)$ is not empty. \\
  We use induction on the size of $\mathcal{B}$ to prove that $C(\mathcal{B})$ is not empty.\\
  Base case: Suppose $\mathcal{B}={x}$, then $C(\mathcal{B})=x$, which is not empty.\\
  Inductive hypothesis: Suppose HARP is true for any $\mathcal{B}$  of size at most $n$.\\
  Now we show the result for sets of size $n+1$.\\
  For any set $\mathcal{A}$ of size $n+1$, we claim that $C(\mathcal{A})=C(\mathcal{A};\succeq)$ \\
  $\mathcal{A}=\{x\}\cup \mathcal{B}$ for some $x \notin \mathcal{B}$ and $\mathcal{B} s.t. |\mathcal{B}|=n$\\
  By inductive hypothesis, we know that $C(\mathcal{B})\neq \phi$\\
  Suppose $y\in C(\mathcal{B})$, because $\succeq$ is complete, one of the following must be true.
  \begin{enumerate}
    \item $x \succeq y$: By transitivity, we must have that $x \succeq z, \forall z \in \mathcal{B}$, so $x\in C(\mathcal{A})$.
    \item $y \succeq x$: Then $y \in C(\mathcal{A})$ by definition.
  \end{enumerate}
  It thus follows by induction that $C(\mathcal{B})=C(\mathcal{B};\succeq)\neq \phi$.\\
  Then we agree that $C(.)$ must satisfy HARP.\\
  $\Longrightarrow$ \\
  If $C(.)$ is non-empty and HARP, we need to show $\exists$ a complete and transitive $\succeq$ s.t. $C(\mathcal{B})=C(\mathcal{B};\succeq), \forall \mathcal{B}$.\\
  First, we construct a preference relation $\succeq$ as follows:\\
  $x \succeq y$ whenever $\exists$ a subset $\mathcal{B}$ s.t. $x, y \in \mathcal{B}$ and $x \in C(\mathcal{B})$.\\
  First we need to show $\succeq$ is complete.\\
  Consider $\mathcal{B}=\{x,y\}$. Since $C(.)$ is non-empty, one of the following must be true:
  \begin{itemize}
  \item $x \in C(\{x,y\}) \Rightarrow x\succeq y$
  \item $y \in C(\{x,y\}) \Rightarrow y\succeq x$
   \item $x, y \in C(\{x,y\}) \Rightarrow y\succeq x , x\succeq y$
  \end{itemize}
  In all cases, we have $y\succeq x$ or $ x\succeq y$, so $\succeq$ is complete.\\
  Then we need to show $\succeq$ is transitive: we need to show that if $x\succeq y$ and $y\succeq z$, then $x\succeq z$.\\
  By definition of $\succeq$, we know that $\exists \mathcal{A}_1$ and $\mathcal{A}_2$ s.t. \\
  $x, y \in \mathcal{A}_1 $ and $x \in C(\mathcal{A}_1)$ \\
    $y, z \in \mathcal{A}_2 $ and $y \in C(\mathcal{A}_2)$ \\
    Let's consider $\mathcal{B}=\{x, y, z\}$, there are three cases.
    \begin{enumerate}
    \item $x \in C(\mathcal{B})$, then by definition, $x \succeq z$
    \item $y \in C(\mathcal{B})$. Now note that $x, y \in \mathcal{A}_1\cap \mathcal{B}, x\in C(\mathcal{A}_1), y \in C(\mathcal{B})$, by HARP, $x \in C(\mathcal{B}) \Longrightarrow x \succeq z$
    \item $z \in C(\mathcal{B})$ Note that $y, z \in \mathcal{A}_2 \cap \mathcal{B}$ and $y \in C(\mathcal{A}_2), z \in C(\mathcal{B})$. By HARP, $y \in C(\mathcal{B})$. And then it reduced to 2.
    \end{enumerate}
Then we need to show $C(.)=C(.;\succeq)$
\begin{enumerate}
\item $C(\mathcal{B}) \subseteq C(\mathcal{B}; \succeq), \forall \mathcal{B} $\\
    $x \in C(\mathcal{B}) $, by definition of $\succeq$, $x\succeq y, \forall y \in \mathcal{B}$. By definition of $C(.;\succeq), x\in C(\mathcal{B};\succeq)  $
\item $C(\mathcal{B}) \supseteq C(\mathcal{B}; \succeq), \forall \mathcal{B} $\\
    $x \in C(\mathcal{B};\succeq)$ Consider an element $y \in C(\mathcal{B})$. Such a $y$ exists because $C(\mathcal{B})\neq \phi$\\
    Because $x \in C(\mathcal{B};\succeq) \Longrightarrow x \succeq y$ .\\
    Since $x \succeq y$, $\exists$ a set $\mathcal{A}$ s.t. $x \in C(\mathcal{A})$ and $x, y \in \mathcal{A}$\\
    Because $y \in C(\mathcal{B})$, we have that $x, y \in \mathcal{A} \cap \mathcal{B}, x \in C(\mathcal{A}), y \in C(\mathcal{B})$, by HARP, $x \in C(\mathcal{B})$.
    \end{enumerate}
    It thus follows that  $C(\mathcal{B}) = C(\mathcal{B}; \succeq), \forall \mathcal{B} $
\end{proof}


\end{document}
